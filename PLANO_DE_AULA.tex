\documentclass[11pt,a4paper]{article}
\usepackage[utf8]{inputenc}
\usepackage[portuguese]{babel}
\usepackage[T1]{fontenc}
\usepackage{geometry}
\usepackage{xcolor}
\usepackage{tcolorbox}
\usepackage{listings}
\usepackage{hyperref}
\usepackage{fancyhdr}
\usepackage{titlesec}
\usepackage{enumitem}
\usepackage{graphicx}
\usepackage{amsmath}
\usepackage{fontawesome5}

% Configuração de página
\geometry{
    left=2.5cm,
    right=2.5cm,
    top=3cm,
    bottom=3cm
}

% Cores personalizadas
\definecolor{primaryblue}{RGB}{41, 128, 185}
\definecolor{secondaryblue}{RGB}{52, 152, 219}
\definecolor{accentgreen}{RGB}{39, 174, 96}
\definecolor{accentorange}{RGB}{230, 126, 34}
\definecolor{accentred}{RGB}{231, 76, 60}
\definecolor{lightgray}{RGB}{236, 240, 241}
\definecolor{darkgray}{RGB}{44, 62, 80}
\definecolor{codebg}{RGB}{248, 249, 250}

% Configuração de código
\lstset{
    backgroundcolor=\color{codebg},
    commentstyle=\color{accentgreen},
    keywordstyle=\color{primaryblue}\bfseries,
    stringstyle=\color{accentred},
    basicstyle=\ttfamily\small,
    breakatwhitespace=false,
    breaklines=true,
    captionpos=b,
    keepspaces=true,
    numbers=left,
    numbersep=5pt,
    showspaces=false,
    showstringspaces=false,
    showtabs=false,
    tabsize=2,
    frame=single,
    frameround=tttt,
    rulecolor=\color{lightgray}
}

% Definição de linguagens
\lstdefinelanguage{GDScript}{
    keywords={extends, func, var, if, elif, else, for, while, return, class, signal, onready, export},
    sensitive=true,
    morecomment=[l]{\#},
    morestring=[b]",
    morestring=[b]'
}

\lstdefinelanguage{CSharp}{
    keywords={using, namespace, class, public, private, protected, static, void, int, float, bool, string, var, if, else, for, while, return, override, new, async, await},
    sensitive=true,
    morecomment=[l]{//},
    morecomment=[s]{/*}{*/},
    morestring=[b]"
}

% Boxes personalizados
\newtcolorbox{theorybox}{
    colback=primaryblue!10,
    colframe=primaryblue,
    title={\faBook\ Teoria},
    fonttitle=\bfseries,
    rounded corners=5pt,
    boxrule=1pt
}

\newtcolorbox{exercisebox}{
    colback=accentgreen!10,
    colframe=accentgreen,
    title={\faDumbbell\ Exercício Prático},
    fonttitle=\bfseries,
    rounded corners=5pt,
    boxrule=1pt
}

\newtcolorbox{tipbox}{
    colback=accentorange!10,
    colframe=accentorange,
    title={\faLightbulb\ Dica},
    fonttitle=\bfseries,
    rounded corners=5pt,
    boxrule=1pt
}

\newtcolorbox{importantbox}{
    colback=accentred!10,
    colframe=accentred,
    title={\faExclamationTriangle\ Importante},
    fonttitle=\bfseries,
    rounded corners=5pt,
    boxrule=1pt
}

\newtcolorbox{checklistbox}{
    colback=secondaryblue!10,
    colframe=secondaryblue,
    title={\faCheckSquare\ Checklist},
    fonttitle=\bfseries,
    rounded corners=5pt,
    boxrule=1pt
}

% Configuração de títulos
\titleformat{\section}
{\Large\bfseries\color{primaryblue}}
{}
{0em}
{}[\titlerule[1pt]]

\titleformat{\subsection}
{\large\bfseries\color{secondaryblue}}
{}
{0em}
{}

\titleformat{\subsubsection}
{\normalsize\bfseries\color{darkgray}}
{}
{0em}
{}

% Cabeçalho e rodapé
\pagestyle{fancy}
\fancyhf{}
\fancyhead[L]{\textcolor{primaryblue}{\textbf{Plano de Aula - Breakout Clone}}}
\fancyhead[R]{\textcolor{darkgray}{Programação de Jogos Digitais - SENAI}}
\fancyfoot[C]{\textcolor{darkgray}{\thepage}}
\renewcommand{\headrulewidth}{0.5pt}
\renewcommand{\footrulewidth}{0pt}

% Configuração de hyperlinks
\hypersetup{
    colorlinks=true,
    linkcolor=primaryblue,
    filecolor=primaryblue,
    urlcolor=primaryblue,
    citecolor=primaryblue,
    pdftitle={Plano de Aula - Breakout Clone},
    pdfauthor={SENAI}
}

% Comandos personalizados
\newcommand{\module}[1]{\section*{\faGraduationCap\ #1}}
\newcommand{\duration}[1]{\textcolor{darkgray}{\textit{Duração: #1}}}

% Remover numeração de seções
\setcounter{secnumdepth}{0}

\begin{document}

% Capa
\begin{titlepage}
    \centering
    \vspace*{2cm}
    
    {\Huge\bfseries\color{primaryblue} Plano de Aula}\\[0.5cm]
    {\LARGE\bfseries Breakout Clone}\\[1cm]
    
    {\Large Programação de Jogos Digitais}\\[0.5cm]
    {\Large SENAI}\\[2cm]
    
    \begin{tcolorbox}[colback=lightgray, colframe=darkgray, width=0.8\textwidth, center]
        \centering
        \textbf{Duração Total:} 4-6 aulas (dependendo do ritmo da turma)\\[0.5cm]
        \textbf{Objetivo:} Desenvolver um jogo completo estilo Breakout usando Godot Engine e GDScript/C\#
    \end{tcolorbox}
    
    \vfill
    
    \begin{tcolorbox}[colback=primaryblue!10, colframe=primaryblue, width=0.8\textwidth, center]
        \centering
        \faGraduationCap\ \textbf{Conteúdo do Curso}\\[0.3cm]
        \begin{itemize}[leftmargin=*, itemsep=0.2cm]
            \item Fundamentos de Programação e Godot Engine
            \item Matemática e Física para Jogos
            \item Input e Controle
            \item Física e Colisões
            \item Interface de Usuário (UI)
            \item Gerenciamento de Estado
            \item Projeto Final: Breakout Clone
        \end{itemize}
    \end{tcolorbox}
    
    \vfill
    
    {\large \today}
\end{titlepage}

\newpage
\tableofcontents
\newpage

% ============================================
% MÓDULO 1
% ============================================
\module{Módulo 1: Fundamentos de Programação e Godot Engine}
\duration{1-2 aulas}

\subsection{Conceitos Fundamentais de Programação}

\begin{theorybox}
\textbf{Variáveis e Tipos de Dados}

As variáveis são espaços na memória que armazenam valores. No desenvolvimento de jogos, trabalhamos com diferentes tipos de dados:

\begin{itemize}
    \item \texttt{int} / \texttt{int}: Números inteiros (ex: 10, -5, 0)
    \item \texttt{float} / \texttt{float}: Números decimais (ex: 3.14, -2.5)
    \item \texttt{bool} / \texttt{bool}: Valores verdadeiro/falso (true/false)
    \item \texttt{String} / \texttt{string}: Texto (ex: "Hello World")
    \item \texttt{Vector2} / \texttt{Vector2}: Par de coordenadas (x, y)
\end{itemize}
\end{theorybox}

\textbf{Exemplo de código:}

\begin{lstlisting}[language=GDScript, caption=GDScript - Declaração de Variáveis]
# GDScript
var score: int = 0
var speed: float = 350.0
var is_moving: bool = false
var player_name: String = "Jogador"
var position: Vector2 = Vector2(100, 200)
\end{lstlisting}

\begin{lstlisting}[language=CSharp, caption=C\# - Declaração de Variáveis]
// C#
int score = 0;
float speed = 350.0f;
bool isMoving = false;
string playerName = "Jogador";
Vector2 position = new Vector2(100, 200);
\end{lstlisting}

\begin{theorybox}
\textbf{Operadores Matemáticos e de Comparação}

\begin{itemize}
    \item \textbf{Operadores Matemáticos:} +, -, *, /, \%, ** (GDScript) ou Mathf.Pow() (C\#)
    \item \textbf{Operadores de Comparação:} ==, !=, >, <, >=, <=
\end{itemize}
\end{theorybox}

\textbf{Estruturas Condicionais:}

\begin{lstlisting}[language=GDScript, caption=GDScript - Condicionais]
if score > 100:
    print("Pontuação alta!")
elif score > 50:
    print("Pontuação média")
else:
    print("Continue tentando")
\end{lstlisting}

\begin{lstlisting}[language=CSharp, caption=C\# - Condicionais]
if (score > 100)
{
    GD.Print("Pontuação alta!");
}
else if (score > 50)
{
    GD.Print("Pontuação média");
}
else
{
    GD.Print("Continue tentando");
}
\end{lstlisting}

\textbf{Loops (Repetições):}

\begin{lstlisting}[language=GDScript, caption=GDScript - Loops]
# Loop for
for i in range(10):
    print(i)  # Imprime 0 a 9

# Loop while
var count = 0
while count < 5:
    print(count)
    count += 1
\end{lstlisting}

\begin{lstlisting}[language=CSharp, caption=C\# - Loops]
// Loop for
for (int i = 0; i < 10; i++)
{
    GD.Print(i);  // Imprime 0 a 9
}

// Loop while
int count = 0;
while (count < 5)
{
    GD.Print(count);
    count++;
}
\end{lstlisting}

\subsubsection{Exercícios Práticos}

\begin{exercisebox}
\textbf{Exercício 1.1: Calculadora Simples}

Crie um script que:
\begin{itemize}
    \item Declare duas variáveis numéricas
    \item Realize as 4 operações básicas
    \item Imprima os resultados
\end{itemize}
\end{exercisebox}

\begin{exercisebox}
\textbf{Exercício 1.2: Verificador de Idade}

Crie um script que:
\begin{itemize}
    \item Receba uma idade
    \item Verifique se é maior de 18 anos
    \item Imprima mensagens diferentes para maior/menor de idade
\end{itemize}
\end{exercisebox}

\begin{exercisebox}
\textbf{Exercício 1.3: Contador}

Crie um script que:
\begin{itemize}
    \item Use um loop para contar de 1 a 100
    \item Imprima apenas números pares
    \item Some todos os números e imprima o total
\end{itemize}
\end{exercisebox}

\subsection{Introdução ao Godot Engine}

\begin{theorybox}
\textbf{O que é o Godot?}

\begin{itemize}
    \item Engine de jogos 2D e 3D gratuita e open-source
    \item Editor visual integrado
    \item Sistema de cenas (Scene System)
    \item Linguagens: GDScript (nativo) e C\# (via .NET)
\end{itemize}
\end{theorybox}

\begin{theorybox}
\textbf{Conceitos Fundamentais do Godot}

\textbf{1. Cenas (Scenes)}
\begin{itemize}
    \item Uma cena é uma árvore de nós (nodes)
    \item Cada elemento do jogo é um nó
    \item Cenas podem ser salvas e reutilizadas
    \item Formato: \texttt{.tscn} (text scene)
\end{itemize}

\textbf{2. Nós (Nodes)}
\begin{itemize}
    \item Unidade básica do Godot
    \item Cada nó tem uma função específica
    \item Nós podem ter filhos (hierarquia)
    \item Exemplos: \texttt{Node}, \texttt{Node2D}, \texttt{CharacterBody2D}, \texttt{StaticBody2D}, \texttt{Label}, \texttt{Button}
\end{itemize}

\textbf{3. Scripts}
\begin{itemize}
    \item Adicionam lógica aos nós
    \item GDScript: \texttt{.gd}
    \item C\#: \texttt{.cs}
    \item Cada nó pode ter um script
\end{itemize}

\textbf{4. Sistema de Coordenadas}
\begin{itemize}
    \item Origem (0, 0) no canto superior esquerdo
    \item Eixo X: aumenta da esquerda para direita
    \item Eixo Y: aumenta de cima para baixo
    \item Unidade: pixels
\end{itemize}
\end{theorybox}

\begin{importantbox}
\textbf{Funções Principais do Godot}

\begin{itemize}
    \item \texttt{\_ready()}: Chamada quando o nó entra na árvore da cena (inicialização)
    \item \texttt{\_process(delta)}: Chamada a cada frame (lógica contínua)
    \item \texttt{\_physics\_process(delta)}: Chamada em intervalos fixos (60 FPS por padrão) - use para física
\end{itemize}
\end{importantbox}

\textbf{Exemplo Básico:}

\begin{lstlisting}[language=GDScript, caption=GDScript - Exemplo Básico]
extends Node2D

var speed: float = 100.0

func _ready():
    print("Nó inicializado!")
    position = Vector2(100, 100)

func _process(delta):
    position.x += speed * delta
\end{lstlisting}

\begin{lstlisting}[language=CSharp, caption=C\# - Exemplo Básico]
using Godot;

public partial class MyNode : Node2D
{
    private float speed = 100.0f;

    public override void _Ready()
    {
        GD.Print("Nó inicializado!");
        Position = new Vector2(100, 100);
    }

    public override void _Process(double delta)
    {
        Position = new Vector2(Position.X + speed * (float)delta, Position.Y);
    }
}
\end{lstlisting}

\subsubsection{Exercícios Práticos}

\begin{exercisebox}
\textbf{Exercício 1.4: Primeira Cena}
\begin{enumerate}
    \item Crie uma nova cena
    \item Adicione um nó \texttt{Node2D} como raiz
    \item Adicione um script ao nó
    \item No \texttt{\_ready()}, imprima "Olá, Godot!"
    \item Execute a cena (F5)
\end{enumerate}
\end{exercisebox}

\begin{exercisebox}
\textbf{Exercício 1.5: Objeto em Movimento}
\begin{enumerate}
    \item Crie uma cena com um \texttt{CharacterBody2D}
    \item Adicione um \texttt{CollisionShape2D} como filho
    \item Configure uma forma de colisão retangular
    \item Adicione um script que mova o objeto da esquerda para direita
    \item Use \texttt{\_physics\_process()} e \texttt{move\_and\_slide()}
\end{enumerate}
\end{exercisebox}

\begin{exercisebox}
\textbf{Exercício 1.6: Contador Visual}
\begin{enumerate}
    \item Crie uma cena com um \texttt{Label}
    \item Adicione um script que conte de 0 a 100
    \item Atualize o texto do label a cada segundo
    \item Use \texttt{\_process()} e controle o tempo com \texttt{delta}
\end{enumerate}
\end{exercisebox}

% ============================================
% MÓDULO 2
% ============================================
\newpage
\module{Módulo 2: Matemática e Física para Jogos}
\duration{1 aula}

\subsection{Vetores e Coordenadas}

\begin{theorybox}
\textbf{O que é um Vetor?}

\begin{itemize}
    \item Representa direção e magnitude
    \item Em 2D: \texttt{Vector2(x, y)}
    \item Usado para posição, velocidade, aceleração
\end{itemize}

\textbf{Operações com Vetores:}

\begin{enumerate}
    \item \textbf{Soma:} \texttt{v1 + v2}
    \item \textbf{Multiplicação por Escalar:} \texttt{vetor * 2}
    \item \textbf{Normalização:} \texttt{vetor.normalized()} (comprimento 1)
    \item \textbf{Comprimento:} \texttt{vetor.length()}
    \item \textbf{Distância:} \texttt{ponto1.distance\_to(ponto2)}
\end{enumerate}
\end{theorybox}

\begin{lstlisting}[language=GDScript, caption=Exemplos de Operações com Vetores]
var v1 = Vector2(10, 20)
var v2 = Vector2(5, -10)
var resultado = v1 + v2  # (15, 10)

var vetor = Vector2(10, 20)
var escalado = vetor * 2  # (20, 40)
var normalizado = vetor.normalized()

var ponto1 = Vector2(0, 0)
var ponto2 = Vector2(3, 4)
var distancia = ponto1.distance_to(ponto2)  # 5.0
\end{lstlisting}

\subsubsection{Exercícios Práticos}

\begin{exercisebox}
\textbf{Exercício 2.1: Calculadora de Vetores}

Crie um script que:
\begin{itemize}
    \item Declare dois vetores
    \item Calcule soma, subtração, multiplicação
    \item Calcule distância entre eles
    \item Normalize um vetor e mostre seu comprimento
\end{itemize}
\end{exercisebox}

\begin{exercisebox}
\textbf{Exercício 2.2: Objeto Seguindo o Mouse}
\begin{enumerate}
    \item Crie um \texttt{CharacterBody2D}
    \item No \texttt{\_process()}, calcule a direção do mouse
    \item Mova o objeto em direção ao mouse
    \item Use \texttt{get\_global\_mouse\_position()}
\end{enumerate}
\end{exercisebox}

\begin{exercisebox}
\textbf{Exercício 2.3: Movimento em Círculo}
\begin{enumerate}
    \item Crie um objeto que se mova em círculo
    \item Use funções trigonométricas: \texttt{sin()} e \texttt{cos()}
    \item Fórmula: \texttt{x = raio * cos(ângulo)}, \texttt{y = raio * sin(ângulo)}
    \item Incremente o ângulo a cada frame
\end{enumerate}
\end{exercisebox}

\subsection{Trigonometria Básica}

\begin{theorybox}
\textbf{Por que precisamos de trigonometria?}

\begin{itemize}
    \item Calcular ângulos de rebatida
    \item Movimento em círculo
    \item Rotação de objetos
    \item Direção de movimento
\end{itemize}

\textbf{Funções Trigonométricas:}

\begin{itemize}
    \item \textbf{Seno (sin):} componente Y
    \item \textbf{Cosseno (cos):} componente X
    \item \textbf{Conversão:} 180° = π radianos, 360° = 2π (TAU)
    \item Funções: \texttt{deg\_to\_rad()} e \texttt{rad\_to\_deg()}
\end{itemize}
\end{theorybox}

\begin{lstlisting}[language=GDScript, caption=Exemplo: Movimento em Direção]
# Movimento em 45 graus
var angle = deg_to_rad(45)  # Converte graus para radianos
var direction = Vector2(cos(angle), sin(angle))
var velocity = direction * speed

# Cálculo de ângulo a partir de vetor
var vetor = Vector2(1, 1)
var angulo = atan2(vetor.y, vetor.x)  # Retorna em radianos
var angulo_graus = rad_to_deg(angulo)  # Converte para graus
\end{lstlisting}

\begin{tipbox}
\textbf{No Breakout:} Usamos trigonometria para calcular o ângulo de rebatida da bola. Quanto mais longe do centro da raquete, mais inclinado o ângulo.
\end{tipbox}

\subsubsection{Exercícios Práticos}

\begin{exercisebox}
\textbf{Exercício 2.4: Calculadora de Ângulos}
\begin{enumerate}
    \item Crie um script que receba um ângulo em graus
    \item Converta para radianos
    \item Calcule seno e cosseno
    \item Crie um vetor direção usando esses valores
    \item Mostre o vetor normalizado
\end{enumerate}
\end{exercisebox}

\begin{exercisebox}
\textbf{Exercício 2.5: Bola com Direção Aleatória}
\begin{enumerate}
    \item Crie uma bola que se move em direção aleatória
    \item Gere um ângulo aleatório entre -45 e 45 graus
    \item Use \texttt{sin()} e \texttt{cos()} para criar o vetor velocidade
    \item Garanta que a bola sempre vá para cima (Y negativo)
\end{enumerate}
\end{exercisebox}

\begin{exercisebox}
\textbf{Exercício 2.6: Rebatida Simples}
\begin{enumerate}
    \item Crie uma bola e uma parede
    \item Quando a bola colidir, calcule o ângulo de reflexão
    \item Use a normal da colisão para refletir a velocidade
    \item Implemente \texttt{velocity.bounce(normal)}
\end{enumerate}
\end{exercisebox}

% ============================================
% MÓDULO 3
% ============================================
\newpage
\module{Módulo 3: Input e Controle}
\duration{1 aula}

\subsection{Sistema de Input do Godot}

\begin{theorybox}
\textbf{Input Actions (Ações de Entrada)}

\begin{itemize}
    \item Configuradas em \textbf{Project Settings > Input Map}
    \item Permitem mapear teclas, botões, etc.
    \item Exemplo: "move\_left", "move\_right", "jump"
\end{itemize}

\textbf{Como Configurar:}
\begin{enumerate}
    \item Vá em \textbf{Project > Project Settings}
    \item Aba \textbf{Input Map}
    \item Adicione nova ação (ex: "move\_left")
    \item Clique no "+" e escolha a tecla
\end{enumerate}
\end{theorybox}

\textbf{Verificando Input no Código:}

\begin{lstlisting}[language=GDScript, caption=GDScript - Input]
# Verificar se tecla está pressionada
if Input.is_action_pressed("move_left"):
    position.x -= speed * delta

# Verificar força da ação (0.0 a 1.0)
var strength = Input.get_action_strength("move_right")
velocity.x = strength * speed

# Verificar se tecla foi pressionada neste frame
if Input.is_action_just_pressed("jump"):
    jump()
\end{lstlisting}

\begin{lstlisting}[language=CSharp, caption=C\# - Input]
// Verificar se tecla está pressionada
if (Input.IsActionPressed("move_left"))
{
    Position = new Vector2(Position.X - speed * (float)delta, Position.Y);
}

// Verificar força da ação (0.0 a 1.0)
float strength = Input.GetActionStrength("move_right");
Velocity = new Vector2(strength * speed, Velocity.Y);

// Verificar se tecla foi pressionada neste frame
if (Input.IsActionJustPressed("jump"))
{
    Jump();
}
\end{lstlisting}

\subsubsection{Exercícios Práticos}

\begin{exercisebox}
\textbf{Exercício 3.1: Objeto Controlado por Teclado}
\begin{enumerate}
    \item Crie um objeto que se move com as setas ou WASD
    \item Configure as ações no Input Map
    \item Use \texttt{get\_action\_strength()} para movimento suave
    \item Limite o movimento dentro da tela
\end{enumerate}
\end{exercisebox}

\begin{exercisebox}
\textbf{Exercício 3.2: Objeto Seguindo o Mouse}
\begin{enumerate}
    \item Crie um objeto que segue o cursor do mouse
    \item Use \texttt{get\_global\_mouse\_position()}
    \item Calcule a direção e mova suavemente
    \item Adicione um limite de velocidade máxima
\end{enumerate}
\end{exercisebox}

\begin{exercisebox}
\textbf{Exercício 3.3: Controles Múltiplos}
\begin{enumerate}
    \item Crie um objeto com múltiplos controles:
    \begin{itemize}
        \item Setas: movimento básico
        \item WASD: movimento alternativo
        \item Mouse: movimento direto
    \end{itemize}
    \item Implemente todos os métodos
    \item Priorize mouse > WASD > Setas
\end{enumerate}
\end{exercisebox}

% ============================================
% MÓDULO 4
% ============================================
\newpage
\module{Módulo 4: Física e Colisões}
\duration{1-2 aulas}

\subsection{Tipos de Corpos Físicos}

\begin{theorybox}
\textbf{CharacterBody2D}
\begin{itemize}
    \item Para objetos controlados pelo jogador
    \item Movimento manual via código
    \item Usa \texttt{move\_and\_slide()} para movimento com colisão
    \item Exemplo: jogador, raquete, bola
\end{itemize}

\textbf{StaticBody2D}
\begin{itemize}
    \item Para objetos imóveis
    \item Não se move, mas pode colidir
    \item Exemplo: paredes, blocos, plataformas
\end{itemize}

\textbf{RigidBody2D}
\begin{itemize}
    \item Para objetos com física automática
    \item Afetado por gravidade e forças
    \item Exemplo: objetos que caem, projéteis
\end{itemize}

\textbf{CollisionShape2D}
\begin{itemize}
    \item Define a forma de colisão
    \item Deve ser filho do corpo físico
    \item Formas: Rectangle, Circle, Capsule, etc.
\end{itemize}
\end{theorybox}

\begin{importantbox}
\textbf{No Breakout:}
\begin{itemize}
    \item Bola: \texttt{CharacterBody2D}
    \item Raquete: \texttt{CharacterBody2D}
    \item Blocos: \texttt{StaticBody2D}
    \item Paredes: \texttt{StaticBody2D}
\end{itemize}
\end{importantbox}

\subsection{Detecção de Colisões}

\begin{theorybox}
\textbf{move\_and\_slide()}
\begin{itemize}
    \item Move o objeto e detecta colisões automaticamente
    \item Retorna \texttt{true} se houve colisão
    \item Armazena informações das colisões
\end{itemize}

\textbf{Normal da Colisão:}
\begin{itemize}
    \item Vetor perpendicular à superfície
    \item Aponta para fora do objeto colidido
    \item Usado para calcular reflexão
\end{itemize}
\end{theorybox}

\textbf{Obtendo Informações de Colisão:}

\begin{lstlisting}[language=GDScript, caption=GDScript - Detecção de Colisão]
func _physics_process(delta):
    move_and_slide()
    
    # Verificar todas as colisões
    for i in get_slide_collision_count():
        var collision = get_slide_collision(i)
        var collider = collision.get_collider()  # Objeto colidido
        var normal = collision.get_normal()      # Normal da colisão
        
        # Verificar tipo do objeto
        if collider.is_in_group("blocks"):
            # Colidiu com um bloco
            collider.queue_free()  # Destrói o bloco
\end{lstlisting}

\begin{lstlisting}[language=CSharp, caption=C\# - Detecção de Colisão]
public override void _PhysicsProcess(double delta)
{
    MoveAndSlide();
    
    // Verificar todas as colisões
    for (int i = 0; i < GetSlideCollisionCount(); i++)
    {
        var collision = GetSlideCollision(i);
        var collider = collision.GetCollider();  // Objeto colidido
        var normal = collision.GetNormal();      // Normal da colisão
        
        // Verificar tipo do objeto
        if (collider is Node node && node.IsInGroup("blocks"))
        {
            // Colidiu com um bloco
            node.QueueFree();  // Destrói o bloco
        }
    }
}
\end{lstlisting}

\textbf{Reflexão (Bounce):}

\begin{lstlisting}[language=GDScript, caption=GDScript - Reflexão]
var reflected = velocity.bounce(normal)
velocity = reflected.normalized() * speed
\end{lstlisting}

\begin{theorybox}
\textbf{Grupos (Groups)}
\begin{itemize}
    \item Permitem identificar objetos por categoria
    \item Adicione: \texttt{add\_to\_group("blocks")} (GDScript) ou \texttt{AddToGroup("blocks")} (C\#)
    \item Verifique: \texttt{is\_in\_group("blocks")} (GDScript) ou \texttt{IsInGroup("blocks")} (C\#)
\end{itemize}

\textbf{No Breakout:} Grupos usados: "paddle", "blocks", "walls", "ceiling", "game\_manager"
\end{theorybox}

\subsubsection{Exercícios Práticos}

\begin{exercisebox}
\textbf{Exercício 4.1: Objeto com Colisão}
\begin{enumerate}
    \item Crie um \texttt{CharacterBody2D}
    \item Adicione \texttt{CollisionShape2D} como filho
    \item Configure uma forma retangular
    \item Adicione movimento e teste colisão com paredes
    \item Use \texttt{move\_and\_slide()}
\end{enumerate}
\end{exercisebox}

\begin{exercisebox}
\textbf{Exercício 4.2: Múltiplos Objetos Colidindo}
\begin{enumerate}
    \item Crie vários objetos estáticos
    \item Crie um objeto dinâmico que colide com eles
    \item Detecte qual objeto foi colidido
    \item Imprima o nome do objeto colidido
\end{enumerate}
\end{exercisebox}

\begin{exercisebox}
\textbf{Exercício 4.3: Bola Rebote Simples}
\begin{enumerate}
    \item Crie uma bola que se move
    \item Crie paredes nas bordas da tela
    \item Quando colidir, reflete a velocidade
    \item Use \texttt{velocity.bounce(normal)}
\end{enumerate}
\end{exercisebox}

\begin{exercisebox}
\textbf{Exercício 4.4: Destruição de Objetos}
\begin{enumerate}
    \item Crie vários blocos estáticos
    \item Adicione-os ao grupo "blocks"
    \item Crie uma bola que, ao colidir, destrói o bloco
    \item Use \texttt{queue\_free()} para destruir
\end{enumerate}
\end{exercisebox}

\begin{exercisebox}
\textbf{Exercício 4.5: Rebatida com Ângulo}
\begin{enumerate}
    \item Crie uma raquete horizontal
    \item Quando a bola colidir, calcule o ângulo baseado na posição de impacto
    \item Quanto mais longe do centro, mais inclinado
    \item Use trigonometria para calcular nova direção
\end{enumerate}
\end{exercisebox}

% ============================================
% MÓDULO 5
% ============================================
\newpage
\module{Módulo 5: Interface de Usuário (UI)}
\duration{1 aula}

\subsection{Criando UI no Godot}

\begin{theorybox}
\textbf{Nós de UI}
\begin{itemize}
    \item \texttt{Control}: Nó base para UI
    \item \texttt{Label}: Exibe texto
    \item \texttt{Button}: Botão clicável
    \item \texttt{VBoxContainer} / \texttt{HBoxContainer}: Organiza elementos
    \item \texttt{MarginContainer}: Adiciona margens
\end{itemize}

\textbf{Criando UI:}
\begin{enumerate}
    \item Adicione um nó \texttt{Control} como container
    \item Adicione elementos filhos (Label, Button, etc.)
    \item Configure posição e tamanho
    \item Conecte sinais (ex: botão pressionado)
\end{enumerate}
\end{theorybox}

\textbf{Acessando Elementos UI:}

\begin{lstlisting}[language=GDScript, caption=GDScript - Acesso a UI]
@onready var score_label: Label = $UI/ScoreLabel
@onready var lives_label: Label = $UI/LivesLabel

func _ready():
    score_label.text = "Pontuação: 0"
    lives_label.text = "Vidas: 3"
\end{lstlisting}

\begin{lstlisting}[language=CSharp, caption=C\# - Acesso a UI]
private Label scoreLabel;
private Label livesLabel;

public override void _Ready()
{
    scoreLabel = GetNode<Label>("UI/ScoreLabel");
    livesLabel = GetNode<Label>("UI/LivesLabel");
    
    scoreLabel.Text = "Pontuação: 0";
    livesLabel.Text = "Vidas: 3";
}
\end{lstlisting}

\begin{tipbox}
\textbf{No Breakout:} UI mostra pontuação, vidas, game over, botão de reiniciar
\end{tipbox}

\subsubsection{Exercícios Práticos}

\begin{exercisebox}
\textbf{Exercício 5.1: Contador Visual}
\begin{enumerate}
    \item Crie uma cena com um \texttt{Label}
    \item Adicione um script que conte de 0 a 100
    \item Atualize o texto do label a cada segundo
    \item Formate o texto: "Contador: 50"
\end{enumerate}
\end{exercisebox}

\begin{exercisebox}
\textbf{Exercício 5.2: Painel de Informações}
\begin{enumerate}
    \item Crie uma UI com:
    \begin{itemize}
        \item Label de pontuação
        \item Label de vidas
        \item Label de tempo
    \end{itemize}
    \item Atualize os valores dinamicamente
    \item Use \texttt{VBoxContainer} para organizar
\end{enumerate}
\end{exercisebox}

\begin{exercisebox}
\textbf{Exercício 5.3: Botão Funcional}
\begin{enumerate}
    \item Crie um botão na UI
    \item Conecte o sinal \texttt{pressed}
    \item Quando clicado, reinicie a cena
    \item Use \texttt{get\_tree().reload\_current\_scene()}
\end{enumerate}
\end{exercisebox}

% ============================================
% MÓDULO 6
% ============================================
\newpage
\module{Módulo 6: Gerenciamento de Estado e Comunicação entre Objetos}
\duration{1 aula}

\subsection{Singleton e GameManager}

\begin{theorybox}
\textbf{O que é um GameManager?}
\begin{itemize}
    \item Objeto que gerencia o estado global do jogo
    \item Controla: pontuação, vidas, fim de jogo
    \item Comunica com outros objetos
\end{itemize}

\textbf{Comunicação entre Objetos:}

\textbf{1. Referência Direta}
\begin{itemize}
    \item Use grupos para encontrar objetos
    \item \texttt{get\_tree().get\_first\_node\_in\_group("game\_manager")}
    \item \texttt{GetTree().GetFirstNodeInGroup("game\_manager")} (C\#)
\end{itemize}

\textbf{2. Grupos}
\begin{itemize}
    \item Organize objetos por categoria
    \item Facilita busca e comunicação
\end{itemize}

\textbf{3. Sinais (Signals)}
\begin{itemize}
    \item Sistema de eventos do Godot
    \item Um objeto emite um sinal, outros escutam
    \item Desacopla objetos
\end{itemize}
\end{theorybox}

\textbf{Exemplo de Comunicação:}

\begin{lstlisting}[language=GDScript, caption=GDScript - GameManager]
# GDScript
var game_manager: Node

func _ready():
    game_manager = get_tree().get_first_node_in_group("game_manager")
    game_manager.on_block_destroyed()
\end{lstlisting}

\begin{lstlisting}[language=CSharp, caption=C\# - GameManager]
// C#
private GameManager gameManager;

public override void _Ready()
{
    gameManager = GetTree().GetFirstNodeInGroup("game_manager") as GameManager;
    gameManager.OnBlockDestroyed();
}
\end{lstlisting}

\begin{tipbox}
\textbf{No Breakout:} GameManager gerencia pontuação e vidas. Bola notifica GameManager quando destrói bloco. GameManager verifica condições de vitória/derrota.
\end{tipbox}

\subsubsection{Exercícios Práticos}

\begin{exercisebox}
\textbf{Exercício 6.1: Sistema de Pontuação}
\begin{enumerate}
    \item Crie um GameManager
    \item Adicione variável de pontuação
    \item Crie um método \texttt{add\_points(points)}
    \item Atualize a UI quando pontos mudarem
\end{enumerate}
\end{exercisebox}

\begin{exercisebox}
\textbf{Exercício 6.2: Comunicação entre Objetos}
\begin{enumerate}
    \item Crie dois objetos: ObjetoA e ObjetoB
    \item ObjetoA deve encontrar ObjetoB usando grupos
    \item ObjetoA chama um método de ObjetoB
    \item ObjetoB responde alterando sua cor
\end{enumerate}
\end{exercisebox}

\begin{exercisebox}
\textbf{Exercício 6.3: Sistema de Vidas}
\begin{enumerate}
    \item Crie um sistema de vidas no GameManager
    \item Quando vida chegar a 0, mostre "Game Over"
    \item Adicione botão de reiniciar
    \item Implemente reinício da cena
\end{enumerate}
\end{exercisebox}

% ============================================
% PROJETO FINAL
% ============================================
\newpage
\module{Projeto Final: Breakout Clone}

\begin{checklistbox}
\textbf{Fase 1: Estrutura Básica}
\begin{itemize}[leftmargin=*]
    \item[$\square$] Criar cena principal (Main.tscn)
    \item[$\square$] Criar cena da bola (Ball.tscn)
    \item[$\square$] Criar cena da raquete (Paddle.tscn)
    \item[$\square$] Criar cena do bloco (Block.tscn)
\end{itemize}
\end{checklistbox}

\begin{checklistbox}
\textbf{Fase 2: Movimento}
\begin{itemize}[leftmargin=*]
    \item[$\square$] Implementar movimento da bola
    \item[$\square$] Implementar controle da raquete
    \item[$\square$] Adicionar limites de tela
\end{itemize}
\end{checklistbox}

\begin{checklistbox}
\textbf{Fase 3: Colisões}
\begin{itemize}[leftmargin=*]
    \item[$\square$] Colisão bola-parede (rebote)
    \item[$\square$] Colisão bola-raquete (rebote com ângulo)
    \item[$\square$] Colisão bola-bloco (destruição)
\end{itemize}
\end{checklistbox}

\begin{checklistbox}
\textbf{Fase 4: Gameplay}
\begin{itemize}[leftmargin=*]
    \item[$\square$] Sistema de pontuação
    \item[$\square$] Sistema de vidas
    \item[$\square$] Criação de blocos
    \item[$\square$] Detecção de vitória/derrota
\end{itemize}
\end{checklistbox}

\begin{checklistbox}
\textbf{Fase 5: UI e Polimento}
\begin{itemize}[leftmargin=*]
    \item[$\square$] Interface de usuário
    \item[$\square$] Tela de game over
    \item[$\square$] Botão de reiniciar
    \item[$\square$] Ajustes de dificuldade
\end{itemize}
\end{checklistbox}

% ============================================
% EXERCÍCIOS DE REVISÃO
% ============================================
\newpage
\section*{Exercícios de Revisão}

\begin{exercisebox}
\textbf{Exercício R1: Mini-Jogo de Rebatida}

Crie um jogo simples onde:
\begin{itemize}
    \item Uma bola se move automaticamente
    \item Uma raquete controlada pelo jogador rebate a bola
    \item Objetivo: manter a bola em jogo o máximo de tempo possível
    \item Mostre tempo de sobrevivência na tela
\end{itemize}
\end{exercisebox}

\begin{exercisebox}
\textbf{Exercício R2: Destruidor de Blocos}

Crie um jogo onde:
\begin{itemize}
    \item Vários blocos são criados na tela
    \item Uma bola destrói os blocos ao colidir
    \item Conte quantos blocos foram destruídos
    \item Mostre mensagem quando todos forem destruídos
\end{itemize}
\end{exercisebox}

\begin{exercisebox}
\textbf{Exercício R3: Sistema Completo}

Combine os exercícios anteriores:
\begin{itemize}
    \item Bola que se move
    \item Raquete controlável
    \item Blocos que são destruídos
    \item Sistema de pontuação
    \item Sistema de vidas
    \item UI completa
\end{itemize}
\end{exercisebox}

% ============================================
% OBJETIVOS E RECURSOS
% ============================================
\newpage
\section*{Objetivos de Aprendizado}

Ao final deste plano de aula, os alunos devem ser capazes de:

\begin{itemize}[leftmargin=*, itemsep=0.3cm]
    \item[\faCheckCircle] Entender conceitos fundamentais de programação
    \item[\faCheckCircle] Trabalhar com o Godot Engine
    \item[\faCheckCircle] Implementar movimento e física
    \item[\faCheckCircle] Detectar e responder a colisões
    \item[\faCheckCircle] Criar interfaces de usuário
    \item[\faCheckCircle] Gerenciar estado do jogo
    \item[\faCheckCircle] Comunicar entre objetos
    \item[\faCheckCircle] Desenvolver um jogo completo do zero
\end{itemize}

\section*{Recursos Adicionais}

\begin{theorybox}
\textbf{Documentação Oficial}
\begin{itemize}
    \item \href{https://docs.godotengine.org/}{Godot Docs}
    \item \href{https://docs.godotengine.org/en/stable/tutorials/scripting/gdscript/index.html}{GDScript Reference}
    \item \href{https://docs.godotengine.org/en/stable/tutorials/scripting/c_sharp/index.html}{C\# Documentation}
\end{itemize}

\textbf{Tutoriais Recomendados}
\begin{itemize}
    \item Godot 2D Game Tutorial (oficial)
    \item Heartbeast Action RPG Tutorial
    \item GDQuest Learn GDScript
\end{itemize}

\textbf{Ferramentas Úteis}
\begin{itemize}
    \item Godot Engine (gratuito)
    \item Visual Studio Code (com extensão Godot)
    \item GIMP ou Photoshop (para sprites, opcional)
\end{itemize}
\end{theorybox}

% ============================================
% CRONOGRAMA E AVALIAÇÃO
% ============================================
\newpage
\section*{Cronograma Sugerido}

\begin{table}[h]
\centering
\begin{tabular}{|p{3cm}|p{11cm}|}
\hline
\textbf{Aula} & \textbf{Conteúdo} \\
\hline
1-2 & \textbf{Módulo 1 (Fundamentos)} \\
    & Teoria: 30 min | Exercícios: 60 min | Revisão: 30 min \\
\hline
3 & \textbf{Módulo 2 (Matemática)} \\
    & Teoria: 30 min | Exercícios: 60 min | Revisão: 30 min \\
\hline
4 & \textbf{Módulo 3 (Input)} \\
    & Teoria: 20 min | Exercícios: 70 min | Revisão: 30 min \\
\hline
5-6 & \textbf{Módulo 4 (Física)} \\
    & Teoria: 40 min | Exercícios: 80 min | Revisão: 30 min \\
\hline
7 & \textbf{Módulo 5 (UI)} \\
    & Teoria: 30 min | Exercícios: 60 min | Revisão: 30 min \\
\hline
8 & \textbf{Módulo 6 (GameManager)} \\
    & Teoria: 30 min | Exercícios: 60 min | Revisão: 30 min \\
\hline
9-10 & \textbf{Projeto Final} \\
      & Implementação: 120 min | Testes e ajustes: 60 min \\
\hline
\end{tabular}
\end{table}

\section*{Avaliação}

\begin{importantbox}
\textbf{Critérios de Avaliação do Projeto Final}

\textbf{Funcionalidade (40\%)}
\begin{itemize}
    \item Bola se move corretamente
    \item Raquete controlável
    \item Colisões funcionam
    \item Blocos são destruídos
    \item Sistema de pontuação
    \item Sistema de vidas
    \item Detecção de vitória/derrota
\end{itemize}

\textbf{Código (30\%)}
\begin{itemize}
    \item Código organizado e comentado
    \item Uso adequado de variáveis e funções
    \item Sem código duplicado
    \item Boas práticas de programação
\end{itemize}

\textbf{Interface (20\%)}
\begin{itemize}
    \item UI funcional e clara
    \item Informações visíveis
    \item Botões funcionais
\end{itemize}

\textbf{Criatividade (10\%)}
\begin{itemize}
    \item Melhorias ou features extras
    \item Visual personalizado
    \item Mecânicas adicionais
\end{itemize}
\end{importantbox}

% ============================================
% RODAPÉ FINAL
% ============================================
\vfill
\begin{center}
\begin{tcolorbox}[colback=primaryblue!10, colframe=primaryblue, width=0.9\textwidth, center]
\centering
\Large\bfseries\color{primaryblue}
Boa sorte com o desenvolvimento! \faRocket

\vspace{0.5cm}
\normalsize
\textit{Este plano de aula foi desenvolvido para o curso de Programação de Jogos Digitais no SENAI.}
\end{tcolorbox}
\end{center}

\end{document}

